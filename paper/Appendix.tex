\begin{section}
\setcounter{equation}{0}
\numberwithin{equation}{section}

We wish to solve expression (\ref{eqRB}) for $R_B=1$ to obtain
expression (\ref{eqtalpha}).  This is equivalent to solving an
expression of the form:
\begin{equation}
\label{eqLnPoly} A t^{-3/2} + \ln t \approx -\gamma + \ln \tau_2,
\end{equation}
where
\begin{equation}
\label{eqA} A = \frac{\ln(\tau_2/\tau_1)}{15 \, Q(\rho /a)
\sqrt{\pi}} \! \Bigg ( \frac{2+\Delta \chi}{\Delta \chi} \Bigg )
(\mu_0 \sigma)^{3/2} a^3.
\end{equation}
Changing the variable $u=t^{-3/2}$, and with some algebra, we can
rewrite Eq. (\ref{eqLnPoly}) as
\begin{equation}
\label{eqVarChange} - \frac{3}{2} A u e^{- \frac{3}{2} A u} \approx
- \frac{3}{2} A e^{\frac{3}{2} (\gamma - \ln \tau_2)}.
\end{equation}
Solutions to an expression of the form $x e^x = C$ are defined as
branches of the  Lambert W function $W[n,C]$, where $n$ are integer
values \cite{Corless1996}. Therefore, the solutions $u_n$ to Eq.
(\ref{eqVarChange}) are
\begin{equation}
u_n \approx - \frac{2}{3A} W \Big [ n, - \frac{3}{2} A
e^{\frac{3}{2}(\gamma - \ln \tau_2)} \Big ]
\end{equation}
We can use Eqs. (\ref{eqtbeta}) and (\ref{eqA}) to
show $A = \frac{2}{3} t_\beta^{3/2}$. By replacing $u_n =
t_n^{-3/2}$:
\begin{align}
&t_n^{-3/2} \approx - t_\beta^{-3/2} W \Big [ n, - t_\beta^{3/2} e^{\frac{3}{2}(\gamma - \ln \tau_2)} \Big ] \nonumber \\
\label{eqWsol}
\implies \; \; &t_n \approx t_\beta \Bigg ( - W \Bigg [ n, - \Big ( \frac{t_\beta}{\tau_2} e^\gamma \Big )^{3/2} \Bigg ] \Bigg )^{-2/3}
\end{align}

Real-valued solutions $W[n,x]$ only exist for
$n=-1,0$ \cite{Corless1996}.  Additionally, for $t_n$ to occur after
the primary field has been removed ($t_n \geq 0$), $W [n,x]$
requires $-1/e \leq x \leq 0$. Thus, by Eq. (\ref{eqWsol}):
\begin{align}
& -1/e \leq - \Big ( \frac{t_\beta}{\tau_2}e^\gamma \Big )^{3/2} \leq 0 \nonumber \\
\label{eqWcond}
\implies & e^{-\frac{2}{3}-\gamma} \approx 0.288267 \geq \frac{t_\beta}{\tau_2} \geq 0
\end{align}

Recall that our choice in after-effect function
(\ref{eqAfterEffectB})  is only valid for $\tau_1 \ll t \ll \tau_2$.
Therefore, the condition defined in expression (\ref{eqWcond}) is
reasonable under the assumption that $t_\beta \ll \tau_2$. We
evaluated Eq. (\ref{eqWsol}) for $n=0$ and noticed the solutions were
$t_0 \not \ll \tau_2$. This violates our conditions for the
after-effect function and is therefore not a valid solution. On the other hand, solutions of Eq.
(\ref{eqWsol}) for $n=-1$ did not violate
conditions for the after-effect function. The solutions
obtained using $W[-1,x]$ consistently showed $t_\alpha \leq
t_\beta$. As a result, the time $t_\alpha$ which solves $R_B=1$ in
expression (\ref{eqRB}) is given by:
%\setcounter{equation}{50}
\begin{equation} t_\alpha \approx  t_\beta \Bigg ( -
W \Bigg [ - \! 1, - \Big  ( \frac{t_\beta}{\tau_2} e^\gamma \Big
)^{3/2} \Bigg ] \Bigg )^{-2/3} \leq t_\beta \tag{47}
\end{equation}
\end{section}
    